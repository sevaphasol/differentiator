\documentclass[12pt, a4paper]{article}
\usepackage[T2A]{fontenc}
\usepackage[utf8x]{inputenc}
\usepackage[english, russian]{babel}
\usepackage{mathtools}
\DeclareMathOperator{\arcsinh}{arcsinh}
\DeclareMathOperator{\arccosh}{arccosh}
\DeclareMathOperator{\arctanh}{arctanh}
\DeclareMathOperator{\arccoth}{arccoth}
\begin{document}
\title{Производная туда сюда
}\author{Севсоль, 1 курс ЭРТЭ}
\date{\today}
\maketitle 
\centerline{Ща производную такой вот функции за яйца возьмём}
\begin{equation}
f(x) = \sqrt{{(x)}^{2}}
\end{equation}
Уважаемая КВМ, пососите мои яйки
\begin{equation}
\frac{d}{dx}(x) = 1
\end{equation}
Методом пристального взгляда
\begin{equation}
\frac{d}{dx}(2) = 0
\end{equation}
Упростим
\begin{equation}
{(x)}^{1} = x
\end{equation}
Согласано предложению 1488 Знаменской Люмдмилы Николаевны
\begin{equation}
\frac{d}{dx}({(x)}^{2}) = 2 \cdot x
\end{equation}
Уважаемая КВМ, пососите мои яйки
\begin{equation}
\frac{d}{dx}(\sqrt{{(x)}^{2}}) = \frac{2 \cdot x}{2 \cdot \sqrt{{(x)}^{2}}}
\end{equation}
Уважаемая КВМ, пососите мои яйки
\begin{equation}
\frac{d}{dx}(2) = 0
\end{equation}
Я устал
\begin{equation}
\frac{d}{dx}(x) = 1
\end{equation}
Упростим
\begin{equation}
0 \cdot x = 0
\end{equation}
Упростим
\begin{equation}
1 \cdot 2 = 2
\end{equation}
Упростим
\begin{equation}
2 = 2
\end{equation}
Упростим
\begin{equation}
2 = 2
\end{equation}
Упростим
\begin{equation}
0+2 = 2
\end{equation}
Упростим
\begin{equation}
2 = 2
\end{equation}
Упростим
\begin{equation}
2 = 2
\end{equation}
Люблю кафедру общесоса
\begin{equation}
\frac{d}{dx}(2 \cdot x) = 2
\end{equation}
Методом пристального взгляда
\begin{equation}
\frac{d}{dx}(2) = 0
\end{equation}
Блять завтра семестровая
\begin{equation}
\frac{d}{dx}(x) = 1
\end{equation}
Согласано предложению 1488 Знаменской Люмдмилы Николаевны
\begin{equation}
\frac{d}{dx}(2) = 0
\end{equation}
Упростим
\begin{equation}
{(x)}^{1} = x
\end{equation}
Я устал
\begin{equation}
\frac{d}{dx}({(x)}^{2}) = 2 \cdot x
\end{equation}
Уважаемая КВМ, пососите мои яйки
\begin{equation}
\frac{d}{dx}(\sqrt{{(x)}^{2}}) = \frac{2 \cdot x}{2 \cdot \sqrt{{(x)}^{2}}}
\end{equation}
Упростим
\begin{equation}
0 \cdot \sqrt{{(x)}^{2}} = 0
\end{equation}
Упростим
\begin{equation}
0+\frac{2 \cdot x}{2 \cdot \sqrt{{(x)}^{2}}} \cdot 2 = -nan
\end{equation}
Уважаемая КВМ, пососите мои яйки
\begin{equation}
\frac{d}{dx}(2 \cdot \sqrt{{(x)}^{2}}) = -nan
\end{equation}
Уважаемая КВМ, пососите мои яйки
\begin{equation}
\frac{d}{dx}(\frac{2 \cdot x}{2 \cdot \sqrt{{(x)}^{2}}}) = \frac{2 \cdot 2 \cdot \sqrt{{(x)}^{2}}--nan \cdot 2 \cdot x}{{(2 \cdot \sqrt{{(x)}^{2}})}^{2}}
\end{equation}
Блять завтра семестровая
\begin{equation}
\frac{d}{dx}(2) = 0
\end{equation}
Вам пора задуматься об обучении на Физтехе
\begin{equation}
\frac{d}{dx}(2) = 0
\end{equation}
Очевидно, что
\begin{equation}
\frac{d}{dx}(x) = 1
\end{equation}
Согласано предложению 1488 Знаменской Люмдмилы Николаевны
\begin{equation}
\frac{d}{dx}(2) = 0
\end{equation}
Упростим
\begin{equation}
{(x)}^{1} = x
\end{equation}
Люблю кафедру общесоса
\begin{equation}
\frac{d}{dx}({(x)}^{2}) = 2 \cdot x
\end{equation}
Я устал
\begin{equation}
\frac{d}{dx}(\sqrt{{(x)}^{2}}) = \frac{2 \cdot x}{2 \cdot \sqrt{{(x)}^{2}}}
\end{equation}
Упростим
\begin{equation}
0 \cdot \sqrt{{(x)}^{2}} = 0
\end{equation}
Упростим
\begin{equation}
0+\frac{2 \cdot x}{2 \cdot \sqrt{{(x)}^{2}}} \cdot 2 = -nan
\end{equation}
Люблю кафедру общесоса
\begin{equation}
\frac{d}{dx}(2 \cdot \sqrt{{(x)}^{2}}) = -nan
\end{equation}
Упростим
\begin{equation}
0 \cdot 2 \cdot \sqrt{{(x)}^{2}} = 0
\end{equation}
Упростим
\begin{equation}
-nan \cdot 2 = -nan
\end{equation}
Упростим
\begin{equation}
0+-nan = -nan
\end{equation}
Упростим
\begin{equation}
-nan = -nan
\end{equation}
Упростим
\begin{equation}
-nan = -nan
\end{equation}
Каждый советский дошкольник знает
\begin{equation}
\frac{d}{dx}(2 \cdot 2 \cdot \sqrt{{(x)}^{2}}) = -nan
\end{equation}
Люблю кафедру общесоса
\begin{equation}
\frac{d}{dx}(-nan) = 0
\end{equation}
Я устал
\begin{equation}
\frac{d}{dx}(2) = 0
\end{equation}
Уважаемая КВМ, пососите мои яйки
\begin{equation}
\frac{d}{dx}(x) = 1
\end{equation}
Упростим
\begin{equation}
0 \cdot x = 0
\end{equation}
Упростим
\begin{equation}
1 \cdot 2 = 2
\end{equation}
Упростим
\begin{equation}
2 = 2
\end{equation}
Упростим
\begin{equation}
2 = 2
\end{equation}
Упростим
\begin{equation}
0+2 = 2
\end{equation}
Упростим
\begin{equation}
2 = 2
\end{equation}
Упростим
\begin{equation}
2 = 2
\end{equation}
Каждый советский дошкольник знает
\begin{equation}
\frac{d}{dx}(2 \cdot x) = 2
\end{equation}
Упростим
\begin{equation}
0 \cdot 2 \cdot x = 0
\end{equation}
Упростим
\begin{equation}
2 \cdot -nan = -nan
\end{equation}
Упростим
\begin{equation}
0+-nan = -nan
\end{equation}
Упростим
\begin{equation}
-nan = -nan
\end{equation}
Упростим
\begin{equation}
-nan = -nan
\end{equation}
Уважаемая КВМ, пососите мои яйки
\begin{equation}
\frac{d}{dx}(-nan \cdot 2 \cdot x) = -nan
\end{equation}
Упростим
\begin{equation}
-nan--nan = -nan
\end{equation}
Очевидно, что
\begin{equation}
\frac{d}{dx}(2 \cdot 2 \cdot \sqrt{{(x)}^{2}}--nan \cdot 2 \cdot x) = -nan
\end{equation}
Люблю кафедру общесоса
\begin{equation}
\frac{d}{dx}(2) = 0
\end{equation}
Уважаемая КВМ, пососите мои яйки
\begin{equation}
\frac{d}{dx}(x) = 1
\end{equation}
Согласано предложению 1488 Знаменской Люмдмилы Николаевны
\begin{equation}
\frac{d}{dx}(2) = 0
\end{equation}
Упростим
\begin{equation}
{(x)}^{1} = x
\end{equation}
Я устал
\begin{equation}
\frac{d}{dx}({(x)}^{2}) = 2 \cdot x
\end{equation}
Каждый советский дошкольник знает
\begin{equation}
\frac{d}{dx}(\sqrt{{(x)}^{2}}) = \frac{2 \cdot x}{2 \cdot \sqrt{{(x)}^{2}}}
\end{equation}
Упростим
\begin{equation}
0 \cdot \sqrt{{(x)}^{2}} = 0
\end{equation}
Упростим
\begin{equation}
0+\frac{2 \cdot x}{2 \cdot \sqrt{{(x)}^{2}}} \cdot 2 = -nan
\end{equation}
Уважаемая КВМ, пососите мои яйки
\begin{equation}
\frac{d}{dx}(2 \cdot \sqrt{{(x)}^{2}}) = -nan
\end{equation}
Каждый советский дошкольник знает
\begin{equation}
\frac{d}{dx}(2) = 0
\end{equation}
Упростим
\begin{equation}
{(2 \cdot \sqrt{{(x)}^{2}})}^{1} = 0
\end{equation}
Упростим
\begin{equation}
2 \cdot 0 = 0
\end{equation}
Упростим
\begin{equation}
0 = 0
\end{equation}
Упростим
\begin{equation}
0 = 0
\end{equation}
Люблю кафедру общесоса
\begin{equation}
\frac{d}{dx}({(2 \cdot \sqrt{{(x)}^{2}})}^{2}) = 0
\end{equation}
Упростим
\begin{equation}
0 \cdot 2 \cdot 2 \cdot \sqrt{{(x)}^{2}}--nan \cdot 2 \cdot x = 0
\end{equation}
Упростим
\begin{equation}
-nan \cdot {(2 \cdot \sqrt{{(x)}^{2}})}^{2}-0 = -nan
\end{equation}
Методом пристального взгляда
\begin{equation}
\frac{d}{dx}(\frac{2 \cdot 2 \cdot \sqrt{{(x)}^{2}}--nan \cdot 2 \cdot x}{{(2 \cdot \sqrt{{(x)}^{2}})}^{2}}) = \frac{-nan}{{({(2 \cdot \sqrt{{(x)}^{2}})}^{2})}^{2}}
\end{equation}


Вот мы и посчитали производную. Кстати, уважаемая КВМ, пососите мои яйки.

\end{document}
