\documentclass[12pt, a4paper]{article}
\usepackage[T2A]{fontenc}
\usepackage[utf8x]{inputenc}
\usepackage[english, russian]{babel}
\usepackage{mathtools}
\DeclareMathOperator{\arcsinh}{arcsinh}
\DeclareMathOperator{\arccosh}{arccosh}
\DeclareMathOperator{\arctanh}{arctanh}
\DeclareMathOperator{\arccoth}{arccoth}
\begin{document}
\title{Производная туда сюда
}\author{Севсоль, 1 курс ЭРТЭ}
\date{\today}
\maketitle Ща проивзодну за яйца возьмём\

\begin{equation}
1+2-1
\end{equation}
Я устал
\begin{equation}
\frac{d}{dx}(1) = 0
\end{equation}
simplify

\begin{equation}
0
\end{equation}
Вам пора задуматься об обучении на Физтехе
\begin{equation}
\frac{d}{dx}(2) = 0
\end{equation}
simplify

\begin{equation}
0
\end{equation}
Вам пора задуматься об обучении на Физтехе
\begin{equation}
\frac{d}{dx}(1) = 0
\end{equation}
simplify

\begin{equation}
0
\end{equation}
Я устал
\begin{equation}
\frac{d}{dx}(2-1) = 0-0
\end{equation}
simplify

\begin{equation}
0-0
\end{equation}
Согласано предложению 1488 Знаменской Люмдмилы Николаевны
\begin{equation}
\frac{d}{dx}(1+2-1) = 0+0-0
\end{equation}
simplify

\begin{equation}
0+0-0
\end{equation}


Вот мы и посчитали производную. Кстати, уважаемая КВМ, пососите мои яйки.

\begin{equation}
0+0-0
\end{equation}

\end{document}
