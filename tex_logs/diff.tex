\documentclass[12pt, a4paper]{article}
\usepackage[T2A]{fontenc}
\usepackage[utf8x]{inputenc}
\usepackage[english, russian]{babel}
\usepackage{mathtools}
\DeclareMathOperator{\arcsinh}{arcsinh}
\DeclareMathOperator{\arccosh}{arccosh}
\DeclareMathOperator{\arctanh}{arctanh}
\DeclareMathOperator{\arccoth}{arccoth}
\begin{document}
\title{Производная туда сюда
}\author{Севсоль, 1 курс ЭРТЭ}
\date{\today}
\maketitle 
\centerline{Разложим в окрестности 0 по формуле тейлора до $\tilde{o}(x^5)$ вот такую функцию} 
\begin{equation}
f(x) = {2}^{x-1}
\end{equation}
\begin{equation}
{f}^{(1)}(x) = {2}^{x-1} \cdot \ln(2)\end{equation}\begin{equation}
{f}^{(2)}(x) = {2}^{x-1} \cdot \ln(2) \cdot \ln(2)\end{equation}\begin{equation}
{f}^{(3)}(x) = {2}^{x-1} \cdot \ln(2) \cdot \ln(2) \cdot \ln(2)\end{equation}\begin{equation}
{f}^{(4)}(x) = {2}^{x-1} \cdot \ln(2) \cdot \ln(2) \cdot \ln(2) \cdot \ln(2)\end{equation}\begin{equation}
{f}^{(5)}(x) = {2}^{x-1} \cdot \ln(2) \cdot \ln(2) \cdot \ln(2) \cdot \ln(2) \cdot \ln(2)\end{equation}\centered{Итоговая формула разложения по Тейлору}
\begin{equation}f(x) = A + \tilde{o}(x^5)
\end{equation}
\end{document}
