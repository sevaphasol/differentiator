\documentclass[12pt, a4paper]{article}
\usepackage[T2A]{fontenc}
\usepackage[utf8x]{inputenc}
\usepackage[english, russian]{babel}
\usepackage{mathtools}
\DeclareMathOperator{\arcsinh}{arcsinh}
\DeclareMathOperator{\arccosh}{arccosh}
\DeclareMathOperator{\arctanh}{arctanh}
\DeclareMathOperator{\arccoth}{arccoth}
\begin{document}
\title{Производная туда сюда
}\author{Севсоль, 1 курс ЭРТЭ}
\date{\today}
\maketitle 
\centerline{Ща производную такой вот функции за яйца возьмём}
\begin{equation}
f(x) = \arctan(x)+\sinh({x}^{2})-\sqrt{{x}^{2}} \cdot \sinh(\cosh(\ln(x)))
\end{equation}
Вам пора задуматься об обучении на Физтехе
\begin{equation}
\frac{d}{dx}(x) = 1
\end{equation}
Методом пристального взгляда
\begin{equation}
\frac{d}{dx}(\arctan(x)) = \frac{1}{1+{x}^{2}}
\end{equation}
Блять завтра семестровая
\begin{equation}
\frac{d}{dx}(x) = 1
\end{equation}
Каждый советский дошкольник знает
\begin{equation}
\frac{d}{dx}(2) = 0
\end{equation}
Блять завтра семестровая
\begin{equation}
\frac{d}{dx}({x}^{2}) = 2 \cdot x
\end{equation}
Методом пристального взгляда
\begin{equation}
\frac{d}{dx}(\sinh({x}^{2})) = \cosh({x}^{2}) \cdot 2 \cdot x
\end{equation}
Вам пора задуматься об обучении на Физтехе
\begin{equation}
\frac{d}{dx}(x) = 1
\end{equation}
Уважаемая КВМ, пососите мои яйки
\begin{equation}
\frac{d}{dx}(\sinh({x}^{2})-x) = \cosh({x}^{2}) \cdot 2 \cdot x-1
\end{equation}
Уважаемая КВМ, пососите мои яйки
\begin{equation}
\frac{d}{dx}(\arctan(x)+\sinh({x}^{2})-x) = \frac{1}{1+{x}^{2}}+\cosh({x}^{2}) \cdot 2 \cdot x-1
\end{equation}
Уважаемая КВМ, пососите мои яйки
\begin{equation}
\frac{d}{dx}(x) = 1
\end{equation}
Люблю кафедру общесоса
\begin{equation}
\frac{d}{dx}(\ln(x)) = \frac{1}{x}
\end{equation}
Вам пора задуматься об обучении на Физтехе
\begin{equation}
\frac{d}{dx}(\cosh(\ln(x))) = \sinh(\ln(x)) \cdot \frac{1}{x}
\end{equation}
Блять завтра семестровая
\begin{equation}
\frac{d}{dx}(\sinh(\cosh(\ln(x)))) = \cosh(\cosh(\ln(x))) \cdot \sinh(\ln(x)) \cdot \frac{1}{x}
\end{equation}
Вам пора задуматься об обучении на Физтехе
\begin{equation}
\frac{d}{dx}(\arctan(x)+\sinh({x}^{2})-x \cdot \sinh(\cosh(\ln(x)))) = \frac{1}{1+{x}^{2}}+\cosh({x}^{2}) \cdot 2 \cdot x-1 \cdot \sinh(\cosh(\ln(x)))+\cosh(\cosh(\ln(x))) \cdot \sinh(\ln(x)) \cdot \frac{1}{x} \cdot \arctan(x)+\sinh({x}^{2})-x
\end{equation}


Вот мы и посчитали производную. Кстати, уважаемая КВМ, пососите мои яйки.

\end{document}
