\documentclass[12pt, a4paper]{article}
\usepackage[T2A]{fontenc}
\usepackage[utf8x]{inputenc}
\usepackage[english, russian]{babel}
\usepackage{mathtools}
\DeclareMathOperator{\arcsinh}{arcsinh}
\DeclareMathOperator{\arccosh}{arccosh}
\DeclareMathOperator{\arctanh}{arctanh}
\DeclareMathOperator{\arccoth}{arccoth}
\begin{document}
\title{Производная туда сюда
}\author{Севсоль, 1 курс ЭРТЭ}
\date{\today}
\maketitle 
\centerline{Ща производную такой вот функции за яйца возьмём}
\begin{equation}
f(x) = \ln(1+x) \cdot \sqrt{1+\sin(x)}
\end{equation}
Блять завтра семестровая
\begin{equation}
\frac{d}{dx}(1) = 0
\end{equation}
Я устал
\begin{equation}
\frac{d}{dx}(x) = 1
\end{equation}
Упростим

\begin{equation}
0+1 = 1
\end{equation}
Упростим

\begin{equation}
1 = 1
\end{equation}
Блять завтра семестровая
\begin{equation}
\frac{d}{dx}(1+x) = 1
\end{equation}
Согласано предложению 1488 Знаменской Люмдмилы Николаевны
\begin{equation}
\frac{d}{dx}(\ln(1+x)) = \frac{1}{1+x}
\end{equation}
Уважаемая КВМ, пососите мои яйки
\begin{equation}
\frac{d}{dx}(1) = 0
\end{equation}
Вам пора задуматься об обучении на Физтехе
\begin{equation}
\frac{d}{dx}(x) = 1
\end{equation}
Блять завтра семестровая
\begin{equation}
\frac{d}{dx}(\sin(x)) = \cos(x)
\end{equation}
Каждый советский дошкольник знает
\begin{equation}
\frac{d}{dx}(1+\sin(x)) = \cos(x)
\end{equation}
Методом пристального взгляда
\begin{equation}
\frac{d}{dx}(\sqrt{1+\sin(x)}) = \frac{\cos(x)}{2 \cdot \sqrt{1+\sin(x)}}
\end{equation}
Люблю кафедру общесоса
\begin{equation}
\frac{d}{dx}(\ln(1+x) \cdot \sqrt{1+\sin(x)}) = A
\end{equation}
$$Где A = \frac{1}{1+x} \cdot \sqrt{1+\sin(x)}+\frac{\cos(x)}{2 \cdot \sqrt{1+\sin(x)}} \cdot \ln(1+x)$$
Очевидно, что
\begin{equation}
\frac{d}{dx}(1) = 0
\end{equation}
Методом пристального взгляда
\begin{equation}
\frac{d}{dx}(1) = 0
\end{equation}
Вам пора задуматься об обучении на Физтехе
\begin{equation}
\frac{d}{dx}(x) = 1
\end{equation}
Упростим

\begin{equation}
0+1 = 1
\end{equation}
Упростим

\begin{equation}
1 = 1
\end{equation}
Люблю кафедру общесоса
\begin{equation}
\frac{d}{dx}(1+x) = 1
\end{equation}
Упростим

\begin{equation}
1 \cdot 1 = 1
\end{equation}
Упростим

\begin{equation}
1 = 1
\end{equation}
Упростим

\begin{equation}
0-1 = -1
\end{equation}
Упростим

\begin{equation}
-1 = -1
\end{equation}
Очевидно, что
\begin{equation}
\frac{d}{dx}(\frac{1}{1+x}) = \frac{-1 \cdot 1}{{(1+x)}^{2}}
\end{equation}
Уважаемая КВМ, пососите мои яйки
\begin{equation}
\frac{d}{dx}(1) = 0
\end{equation}
Люблю кафедру общесоса
\begin{equation}
\frac{d}{dx}(x) = 1
\end{equation}
Методом пристального взгляда
\begin{equation}
\frac{d}{dx}(\sin(x)) = \cos(x)
\end{equation}
Люблю кафедру общесоса
\begin{equation}
\frac{d}{dx}(1+\sin(x)) = \cos(x)
\end{equation}
Блять завтра семестровая
\begin{equation}
\frac{d}{dx}(\sqrt{1+\sin(x)}) = \frac{\cos(x)}{2 \cdot \sqrt{1+\sin(x)}}
\end{equation}
Упростим

\begin{equation}
-1 \cdot 1 = -1
\end{equation}
Упростим

\begin{equation}
-1 = -1
\end{equation}
Вам пора задуматься об обучении на Физтехе
\begin{equation}
\frac{d}{dx}(\frac{1}{1+x} \cdot \sqrt{1+\sin(x)}) = B
\end{equation}
$$Где B = \frac{-1}{{(1+x)}^{2}} \cdot \sqrt{1+\sin(x)}+\frac{\cos(x)}{2 \cdot \sqrt{1+\sin(x)}} \cdot \frac{1}{1+x}$$
Согласано предложению 1488 Знаменской Люмдмилы Николаевны
\begin{equation}
\frac{d}{dx}(x) = 1
\end{equation}
Очевидно, что
\begin{equation}
\frac{d}{dx}(\cos(x)) = -1 \cdot \sin(x)
\end{equation}
Я устал
\begin{equation}
\frac{d}{dx}(2) = 0
\end{equation}
Очевидно, что
\begin{equation}
\frac{d}{dx}(1) = 0
\end{equation}
Блять завтра семестровая
\begin{equation}
\frac{d}{dx}(x) = 1
\end{equation}
Согласано предложению 1488 Знаменской Люмдмилы Николаевны
\begin{equation}
\frac{d}{dx}(\sin(x)) = \cos(x)
\end{equation}
Согласано предложению 1488 Знаменской Люмдмилы Николаевны
\begin{equation}
\frac{d}{dx}(1+\sin(x)) = \cos(x)
\end{equation}
Уважаемая КВМ, пососите мои яйки
\begin{equation}
\frac{d}{dx}(\sqrt{1+\sin(x)}) = \frac{\cos(x)}{2 \cdot \sqrt{1+\sin(x)}}
\end{equation}
Люблю кафедру общесоса
\begin{equation}
\frac{d}{dx}(2 \cdot \sqrt{1+\sin(x)}) = \frac{\cos(x)}{2 \cdot \sqrt{1+\sin(x)}} \cdot 2
\end{equation}
Уважаемая КВМ, пососите мои яйки
\begin{equation}
\frac{d}{dx}(\frac{\cos(x)}{2 \cdot \sqrt{1+\sin(x)}}) = D
\end{equation}
$$Где D = \frac{C}{{(2 \cdot \sqrt{1+\sin(x)})}^{2}}$$
$$Где C = -1 \cdot \sin(x) \cdot 2 \cdot \sqrt{1+\sin(x)}-\frac{\cos(x)}{2 \cdot \sqrt{1+\sin(x)}} \cdot 2 \cdot \cos(x)$$
Каждый советский дошкольник знает
\begin{equation}
\frac{d}{dx}(1) = 0
\end{equation}
Каждый советский дошкольник знает
\begin{equation}
\frac{d}{dx}(x) = 1
\end{equation}
Упростим

\begin{equation}
0+1 = 1
\end{equation}
Упростим

\begin{equation}
1 = 1
\end{equation}
Методом пристального взгляда
\begin{equation}
\frac{d}{dx}(1+x) = 1
\end{equation}
Согласано предложению 1488 Знаменской Люмдмилы Николаевны
\begin{equation}
\frac{d}{dx}(\ln(1+x)) = \frac{1}{1+x}
\end{equation}
Уважаемая КВМ, пососите мои яйки
\begin{equation}
\frac{d}{dx}(\frac{\cos(x)}{2 \cdot \sqrt{1+\sin(x)}} \cdot \ln(1+x)) = E
\end{equation}
$$Где E = D \cdot \ln(1+x)+\frac{1}{1+x} \cdot \frac{\cos(x)}{2 \cdot \sqrt{1+\sin(x)}}$$
$$Где D = \frac{C}{{(2 \cdot \sqrt{1+\sin(x)})}^{2}}$$
$$Где C = -1 \cdot \sin(x) \cdot 2 \cdot \sqrt{1+\sin(x)}-\frac{\cos(x)}{2 \cdot \sqrt{1+\sin(x)}} \cdot 2 \cdot \cos(x)$$
Вам пора задуматься об обучении на Физтехе
\begin{equation}
\frac{d}{dx}(A) = B+E
\end{equation}
$$Где A = \frac{1}{1+x} \cdot \sqrt{1+\sin(x)}+\frac{\cos(x)}{2 \cdot \sqrt{1+\sin(x)}} \cdot \ln(1+x)$$
$$Где B = \frac{-1}{{(1+x)}^{2}} \cdot \sqrt{1+\sin(x)}+\frac{\cos(x)}{2 \cdot \sqrt{1+\sin(x)}} \cdot \frac{1}{1+x}$$
$$Где E = D \cdot \ln(1+x)+\frac{1}{1+x} \cdot \frac{\cos(x)}{2 \cdot \sqrt{1+\sin(x)}}$$
$$Где D = \frac{C}{{(2 \cdot \sqrt{1+\sin(x)})}^{2}}$$
$$Где C = -1 \cdot \sin(x) \cdot 2 \cdot \sqrt{1+\sin(x)}-\frac{\cos(x)}{2 \cdot \sqrt{1+\sin(x)}} \cdot 2 \cdot \cos(x)$$
Каждый советский дошкольник знает
\begin{equation}
\frac{d}{dx}(-1) = 0
\end{equation}
Вам пора задуматься об обучении на Физтехе
\begin{equation}
\frac{d}{dx}(1) = 0
\end{equation}
Люблю кафедру общесоса
\begin{equation}
\frac{d}{dx}(x) = 1
\end{equation}
Упростим

\begin{equation}
0+1 = 1
\end{equation}
Упростим

\begin{equation}
1 = 1
\end{equation}
Блять завтра семестровая
\begin{equation}
\frac{d}{dx}(1+x) = 1
\end{equation}
Вам пора задуматься об обучении на Физтехе
\begin{equation}
\frac{d}{dx}(2) = 0
\end{equation}
Люблю кафедру общесоса
\begin{equation}
\frac{d}{dx}({(1+x)}^{2}) = 2 \cdot 1+x
\end{equation}
Уважаемая КВМ, пососите мои яйки
\begin{equation}
\frac{d}{dx}(\frac{-1}{{(1+x)}^{2}}) = \frac{-1 \cdot 2 \cdot 1+x \cdot -1}{{({(1+x)}^{2})}^{2}}
\end{equation}
Очевидно, что
\begin{equation}
\frac{d}{dx}(1) = 0
\end{equation}
Люблю кафедру общесоса
\begin{equation}
\frac{d}{dx}(x) = 1
\end{equation}
Согласано предложению 1488 Знаменской Люмдмилы Николаевны
\begin{equation}
\frac{d}{dx}(\sin(x)) = \cos(x)
\end{equation}
Вам пора задуматься об обучении на Физтехе
\begin{equation}
\frac{d}{dx}(1+\sin(x)) = \cos(x)
\end{equation}
Вам пора задуматься об обучении на Физтехе
\begin{equation}
\frac{d}{dx}(\sqrt{1+\sin(x)}) = \frac{\cos(x)}{2 \cdot \sqrt{1+\sin(x)}}
\end{equation}
Очевидно, что
\begin{equation}
\frac{d}{dx}(\frac{-1}{{(1+x)}^{2}} \cdot \sqrt{1+\sin(x)}) = F
\end{equation}
$$Где F = \frac{-1 \cdot 2 \cdot 1+x \cdot -1}{{({(1+x)}^{2})}^{2}} \cdot \sqrt{1+\sin(x)}+\frac{\cos(x)}{2 \cdot \sqrt{1+\sin(x)}} \cdot \frac{-1}{{(1+x)}^{2}}$$
Очевидно, что
\begin{equation}
\frac{d}{dx}(x) = 1
\end{equation}
Я устал
\begin{equation}
\frac{d}{dx}(\cos(x)) = -1 \cdot \sin(x)
\end{equation}
Согласано предложению 1488 Знаменской Люмдмилы Николаевны
\begin{equation}
\frac{d}{dx}(2) = 0
\end{equation}
Очевидно, что
\begin{equation}
\frac{d}{dx}(1) = 0
\end{equation}
Блять завтра семестровая
\begin{equation}
\frac{d}{dx}(x) = 1
\end{equation}
Люблю кафедру общесоса
\begin{equation}
\frac{d}{dx}(\sin(x)) = \cos(x)
\end{equation}
Блять завтра семестровая
\begin{equation}
\frac{d}{dx}(1+\sin(x)) = \cos(x)
\end{equation}
Люблю кафедру общесоса
\begin{equation}
\frac{d}{dx}(\sqrt{1+\sin(x)}) = \frac{\cos(x)}{2 \cdot \sqrt{1+\sin(x)}}
\end{equation}
Очевидно, что
\begin{equation}
\frac{d}{dx}(2 \cdot \sqrt{1+\sin(x)}) = \frac{\cos(x)}{2 \cdot \sqrt{1+\sin(x)}} \cdot 2
\end{equation}
Каждый советский дошкольник знает
\begin{equation}
\frac{d}{dx}(\frac{\cos(x)}{2 \cdot \sqrt{1+\sin(x)}}) = H
\end{equation}
$$Где H = \frac{G}{{(2 \cdot \sqrt{1+\sin(x)})}^{2}}$$
$$Где G = -1 \cdot \sin(x) \cdot 2 \cdot \sqrt{1+\sin(x)}-\frac{\cos(x)}{2 \cdot \sqrt{1+\sin(x)}} \cdot 2 \cdot \cos(x)$$
Я устал
\begin{equation}
\frac{d}{dx}(1) = 0
\end{equation}
Уважаемая КВМ, пососите мои яйки
\begin{equation}
\frac{d}{dx}(1) = 0
\end{equation}
Каждый советский дошкольник знает
\begin{equation}
\frac{d}{dx}(x) = 1
\end{equation}
Упростим

\begin{equation}
0+1 = 1
\end{equation}
Упростим

\begin{equation}
1 = 1
\end{equation}
Я устал
\begin{equation}
\frac{d}{dx}(1+x) = 1
\end{equation}
Упростим

\begin{equation}
1 \cdot 1 = 1
\end{equation}
Упростим

\begin{equation}
1 = 1
\end{equation}
Упростим

\begin{equation}
0-1 = -1
\end{equation}
Упростим

\begin{equation}
-1 = -1
\end{equation}
Уважаемая КВМ, пососите мои яйки
\begin{equation}
\frac{d}{dx}(\frac{1}{1+x}) = \frac{-1 \cdot 1}{{(1+x)}^{2}}
\end{equation}
Упростим

\begin{equation}
-1 \cdot 1 = -1
\end{equation}
Упростим

\begin{equation}
-1 = -1
\end{equation}
Согласано предложению 1488 Знаменской Люмдмилы Николаевны
\begin{equation}
\frac{d}{dx}(\frac{\cos(x)}{2 \cdot \sqrt{1+\sin(x)}} \cdot \frac{1}{1+x}) = H \cdot \frac{1}{1+x}+\frac{-1}{{(1+x)}^{2}} \cdot \frac{\cos(x)}{2 \cdot \sqrt{1+\sin(x)}}
\end{equation}
$$Где H = \frac{G}{{(2 \cdot \sqrt{1+\sin(x)})}^{2}}$$
$$Где G = -1 \cdot \sin(x) \cdot 2 \cdot \sqrt{1+\sin(x)}-\frac{\cos(x)}{2 \cdot \sqrt{1+\sin(x)}} \cdot 2 \cdot \cos(x)$$
Каждый советский дошкольник знает
\begin{equation}
\frac{d}{dx}(B) = F+H \cdot \frac{1}{1+x}+\frac{-1}{{(1+x)}^{2}} \cdot \frac{\cos(x)}{2 \cdot \sqrt{1+\sin(x)}}
\end{equation}
$$Где B = \frac{-1}{{(1+x)}^{2}} \cdot \sqrt{1+\sin(x)}+\frac{\cos(x)}{2 \cdot \sqrt{1+\sin(x)}} \cdot \frac{1}{1+x}$$
$$Где F = \frac{-1 \cdot 2 \cdot 1+x \cdot -1}{{({(1+x)}^{2})}^{2}} \cdot \sqrt{1+\sin(x)}+\frac{\cos(x)}{2 \cdot \sqrt{1+\sin(x)}} \cdot \frac{-1}{{(1+x)}^{2}}$$
$$Где H = \frac{G}{{(2 \cdot \sqrt{1+\sin(x)})}^{2}}$$
$$Где G = -1 \cdot \sin(x) \cdot 2 \cdot \sqrt{1+\sin(x)}-\frac{\cos(x)}{2 \cdot \sqrt{1+\sin(x)}} \cdot 2 \cdot \cos(x)$$
Блять завтра семестровая
\begin{equation}
\frac{d}{dx}(-1) = 0
\end{equation}
Я устал
\begin{equation}
\frac{d}{dx}(x) = 1
\end{equation}
Согласано предложению 1488 Знаменской Люмдмилы Николаевны
\begin{equation}
\frac{d}{dx}(\sin(x)) = \cos(x)
\end{equation}
Методом пристального взгляда
\begin{equation}
\frac{d}{dx}(-1 \cdot \sin(x)) = \cos(x) \cdot -1
\end{equation}
Согласано предложению 1488 Знаменской Люмдмилы Николаевны
\begin{equation}
\frac{d}{dx}(2) = 0
\end{equation}
Методом пристального взгляда
\begin{equation}
\frac{d}{dx}(1) = 0
\end{equation}
Согласано предложению 1488 Знаменской Люмдмилы Николаевны
\begin{equation}
\frac{d}{dx}(x) = 1
\end{equation}
Блять завтра семестровая
\begin{equation}
\frac{d}{dx}(\sin(x)) = \cos(x)
\end{equation}
Очевидно, что
\begin{equation}
\frac{d}{dx}(1+\sin(x)) = \cos(x)
\end{equation}
Каждый советский дошкольник знает
\begin{equation}
\frac{d}{dx}(\sqrt{1+\sin(x)}) = \frac{\cos(x)}{2 \cdot \sqrt{1+\sin(x)}}
\end{equation}
Вам пора задуматься об обучении на Физтехе
\begin{equation}
\frac{d}{dx}(2 \cdot \sqrt{1+\sin(x)}) = \frac{\cos(x)}{2 \cdot \sqrt{1+\sin(x)}} \cdot 2
\end{equation}
Каждый советский дошкольник знает
\begin{equation}
\frac{d}{dx}(-1 \cdot \sin(x) \cdot 2 \cdot \sqrt{1+\sin(x)}) = I
\end{equation}
$$Где I = \cos(x) \cdot -1 \cdot 2 \cdot \sqrt{1+\sin(x)}+\frac{\cos(x)}{2 \cdot \sqrt{1+\sin(x)}} \cdot 2 \cdot -1 \cdot \sin(x)$$
Согласано предложению 1488 Знаменской Люмдмилы Николаевны
\begin{equation}
\frac{d}{dx}(x) = 1
\end{equation}
Уважаемая КВМ, пососите мои яйки
\begin{equation}
\frac{d}{dx}(\cos(x)) = -1 \cdot \sin(x)
\end{equation}
Каждый советский дошкольник знает
\begin{equation}
\frac{d}{dx}(2) = 0
\end{equation}
Вам пора задуматься об обучении на Физтехе
\begin{equation}
\frac{d}{dx}(1) = 0
\end{equation}
Очевидно, что
\begin{equation}
\frac{d}{dx}(x) = 1
\end{equation}
Очевидно, что
\begin{equation}
\frac{d}{dx}(\sin(x)) = \cos(x)
\end{equation}
Каждый советский дошкольник знает
\begin{equation}
\frac{d}{dx}(1+\sin(x)) = \cos(x)
\end{equation}
Уважаемая КВМ, пососите мои яйки
\begin{equation}
\frac{d}{dx}(\sqrt{1+\sin(x)}) = \frac{\cos(x)}{2 \cdot \sqrt{1+\sin(x)}}
\end{equation}
Люблю кафедру общесоса
\begin{equation}
\frac{d}{dx}(2 \cdot \sqrt{1+\sin(x)}) = \frac{\cos(x)}{2 \cdot \sqrt{1+\sin(x)}} \cdot 2
\end{equation}
Люблю кафедру общесоса
\begin{equation}
\frac{d}{dx}(\frac{\cos(x)}{2 \cdot \sqrt{1+\sin(x)}}) = K
\end{equation}
$$Где K = \frac{J}{{(2 \cdot \sqrt{1+\sin(x)})}^{2}}$$
$$Где J = -1 \cdot \sin(x) \cdot 2 \cdot \sqrt{1+\sin(x)}-\frac{\cos(x)}{2 \cdot \sqrt{1+\sin(x)}} \cdot 2 \cdot \cos(x)$$
Вам пора задуматься об обучении на Физтехе
\begin{equation}
\frac{d}{dx}(2) = 0
\end{equation}
Люблю кафедру общесоса
\begin{equation}
\frac{d}{dx}(\frac{\cos(x)}{2 \cdot \sqrt{1+\sin(x)}} \cdot 2) = K \cdot 2
\end{equation}
$$Где K = \frac{J}{{(2 \cdot \sqrt{1+\sin(x)})}^{2}}$$
$$Где J = -1 \cdot \sin(x) \cdot 2 \cdot \sqrt{1+\sin(x)}-\frac{\cos(x)}{2 \cdot \sqrt{1+\sin(x)}} \cdot 2 \cdot \cos(x)$$
Каждый советский дошкольник знает
\begin{equation}
\frac{d}{dx}(x) = 1
\end{equation}
Люблю кафедру общесоса
\begin{equation}
\frac{d}{dx}(\cos(x)) = -1 \cdot \sin(x)
\end{equation}
Блять завтра семестровая
\begin{equation}
\frac{d}{dx}(\frac{\cos(x)}{2 \cdot \sqrt{1+\sin(x)}} \cdot 2 \cdot \cos(x)) = L
\end{equation}
$$Где L = K \cdot 2 \cdot \cos(x)+-1 \cdot \sin(x) \cdot \frac{\cos(x)}{2 \cdot \sqrt{1+\sin(x)}} \cdot 2$$
$$Где K = \frac{J}{{(2 \cdot \sqrt{1+\sin(x)})}^{2}}$$
$$Где J = -1 \cdot \sin(x) \cdot 2 \cdot \sqrt{1+\sin(x)}-\frac{\cos(x)}{2 \cdot \sqrt{1+\sin(x)}} \cdot 2 \cdot \cos(x)$$
Методом пристального взгляда
\begin{equation}
\frac{d}{dx}(C) = I-L
\end{equation}
$$Где C = -1 \cdot \sin(x) \cdot 2 \cdot \sqrt{1+\sin(x)}-\frac{\cos(x)}{2 \cdot \sqrt{1+\sin(x)}} \cdot 2 \cdot \cos(x)$$
$$Где I = \cos(x) \cdot -1 \cdot 2 \cdot \sqrt{1+\sin(x)}+\frac{\cos(x)}{2 \cdot \sqrt{1+\sin(x)}} \cdot 2 \cdot -1 \cdot \sin(x)$$
$$Где L = K \cdot 2 \cdot \cos(x)+-1 \cdot \sin(x) \cdot \frac{\cos(x)}{2 \cdot \sqrt{1+\sin(x)}} \cdot 2$$
$$Где K = \frac{J}{{(2 \cdot \sqrt{1+\sin(x)})}^{2}}$$
$$Где J = -1 \cdot \sin(x) \cdot 2 \cdot \sqrt{1+\sin(x)}-\frac{\cos(x)}{2 \cdot \sqrt{1+\sin(x)}} \cdot 2 \cdot \cos(x)$$
Я устал
\begin{equation}
\frac{d}{dx}(2) = 0
\end{equation}
Вам пора задуматься об обучении на Физтехе
\begin{equation}
\frac{d}{dx}(1) = 0
\end{equation}
Вам пора задуматься об обучении на Физтехе
\begin{equation}
\frac{d}{dx}(x) = 1
\end{equation}
Согласано предложению 1488 Знаменской Люмдмилы Николаевны
\begin{equation}
\frac{d}{dx}(\sin(x)) = \cos(x)
\end{equation}
Методом пристального взгляда
\begin{equation}
\frac{d}{dx}(1+\sin(x)) = \cos(x)
\end{equation}
Я устал
\begin{equation}
\frac{d}{dx}(\sqrt{1+\sin(x)}) = \frac{\cos(x)}{2 \cdot \sqrt{1+\sin(x)}}
\end{equation}
Согласано предложению 1488 Знаменской Люмдмилы Николаевны
\begin{equation}
\frac{d}{dx}(2 \cdot \sqrt{1+\sin(x)}) = \frac{\cos(x)}{2 \cdot \sqrt{1+\sin(x)}} \cdot 2
\end{equation}
Я устал
\begin{equation}
\frac{d}{dx}(2) = 0
\end{equation}
Я устал
\begin{equation}
\frac{d}{dx}({(2 \cdot \sqrt{1+\sin(x)})}^{2}) = 2 \cdot 2 \cdot \sqrt{1+\sin(x)}
\end{equation}
Уважаемая КВМ, пососите мои яйки
\begin{equation}
\frac{d}{dx}(D) = N
\end{equation}
$$Где D = \frac{C}{{(2 \cdot \sqrt{1+\sin(x)})}^{2}}$$
$$Где C = -1 \cdot \sin(x) \cdot 2 \cdot \sqrt{1+\sin(x)}-\frac{\cos(x)}{2 \cdot \sqrt{1+\sin(x)}} \cdot 2 \cdot \cos(x)$$
$$Где N = \frac{M}{{({(2 \cdot \sqrt{1+\sin(x)})}^{2})}^{2}}$$
$$Где M = I-L \cdot {(2 \cdot \sqrt{1+\sin(x)})}^{2}-2 \cdot 2 \cdot \sqrt{1+\sin(x)} \cdot C$$
$$Где I = \cos(x) \cdot -1 \cdot 2 \cdot \sqrt{1+\sin(x)}+\frac{\cos(x)}{2 \cdot \sqrt{1+\sin(x)}} \cdot 2 \cdot -1 \cdot \sin(x)$$
$$Где L = K \cdot 2 \cdot \cos(x)+-1 \cdot \sin(x) \cdot \frac{\cos(x)}{2 \cdot \sqrt{1+\sin(x)}} \cdot 2$$
$$Где K = \frac{J}{{(2 \cdot \sqrt{1+\sin(x)})}^{2}}$$
$$Где J = -1 \cdot \sin(x) \cdot 2 \cdot \sqrt{1+\sin(x)}-\frac{\cos(x)}{2 \cdot \sqrt{1+\sin(x)}} \cdot 2 \cdot \cos(x)$$
$$Где C = -1 \cdot \sin(x) \cdot 2 \cdot \sqrt{1+\sin(x)}-\frac{\cos(x)}{2 \cdot \sqrt{1+\sin(x)}} \cdot 2 \cdot \cos(x)$$
Вам пора задуматься об обучении на Физтехе
\begin{equation}
\frac{d}{dx}(1) = 0
\end{equation}
Согласано предложению 1488 Знаменской Люмдмилы Николаевны
\begin{equation}
\frac{d}{dx}(x) = 1
\end{equation}
Упростим

\begin{equation}
0+1 = 1
\end{equation}
Упростим

\begin{equation}
1 = 1
\end{equation}
Вам пора задуматься об обучении на Физтехе
\begin{equation}
\frac{d}{dx}(1+x) = 1
\end{equation}
Блять завтра семестровая
\begin{equation}
\frac{d}{dx}(\ln(1+x)) = \frac{1}{1+x}
\end{equation}
Я устал
\begin{equation}
\frac{d}{dx}(D \cdot \ln(1+x)) = N \cdot \ln(1+x)+\frac{1}{1+x} \cdot D
\end{equation}
$$Где D = \frac{C}{{(2 \cdot \sqrt{1+\sin(x)})}^{2}}$$
$$Где C = -1 \cdot \sin(x) \cdot 2 \cdot \sqrt{1+\sin(x)}-\frac{\cos(x)}{2 \cdot \sqrt{1+\sin(x)}} \cdot 2 \cdot \cos(x)$$
$$Где N = \frac{M}{{({(2 \cdot \sqrt{1+\sin(x)})}^{2})}^{2}}$$
$$Где M = I-L \cdot {(2 \cdot \sqrt{1+\sin(x)})}^{2}-2 \cdot 2 \cdot \sqrt{1+\sin(x)} \cdot C$$
$$Где I = \cos(x) \cdot -1 \cdot 2 \cdot \sqrt{1+\sin(x)}+\frac{\cos(x)}{2 \cdot \sqrt{1+\sin(x)}} \cdot 2 \cdot -1 \cdot \sin(x)$$
$$Где L = K \cdot 2 \cdot \cos(x)+-1 \cdot \sin(x) \cdot \frac{\cos(x)}{2 \cdot \sqrt{1+\sin(x)}} \cdot 2$$
$$Где K = \frac{J}{{(2 \cdot \sqrt{1+\sin(x)})}^{2}}$$
$$Где J = -1 \cdot \sin(x) \cdot 2 \cdot \sqrt{1+\sin(x)}-\frac{\cos(x)}{2 \cdot \sqrt{1+\sin(x)}} \cdot 2 \cdot \cos(x)$$
$$Где C = -1 \cdot \sin(x) \cdot 2 \cdot \sqrt{1+\sin(x)}-\frac{\cos(x)}{2 \cdot \sqrt{1+\sin(x)}} \cdot 2 \cdot \cos(x)$$
$$Где D = \frac{C}{{(2 \cdot \sqrt{1+\sin(x)})}^{2}}$$
$$Где C = -1 \cdot \sin(x) \cdot 2 \cdot \sqrt{1+\sin(x)}-\frac{\cos(x)}{2 \cdot \sqrt{1+\sin(x)}} \cdot 2 \cdot \cos(x)$$
Я устал
\begin{equation}
\frac{d}{dx}(1) = 0
\end{equation}
Методом пристального взгляда
\begin{equation}
\frac{d}{dx}(1) = 0
\end{equation}
Я устал
\begin{equation}
\frac{d}{dx}(x) = 1
\end{equation}
Упростим

\begin{equation}
0+1 = 1
\end{equation}
Упростим

\begin{equation}
1 = 1
\end{equation}
Уважаемая КВМ, пососите мои яйки
\begin{equation}
\frac{d}{dx}(1+x) = 1
\end{equation}
Упростим

\begin{equation}
1 \cdot 1 = 1
\end{equation}
Упростим

\begin{equation}
1 = 1
\end{equation}
Упростим

\begin{equation}
0-1 = -1
\end{equation}
Упростим

\begin{equation}
-1 = -1
\end{equation}
Я устал
\begin{equation}
\frac{d}{dx}(\frac{1}{1+x}) = \frac{-1 \cdot 1}{{(1+x)}^{2}}
\end{equation}
Я устал
\begin{equation}
\frac{d}{dx}(x) = 1
\end{equation}
Вам пора задуматься об обучении на Физтехе
\begin{equation}
\frac{d}{dx}(\cos(x)) = -1 \cdot \sin(x)
\end{equation}
Уважаемая КВМ, пососите мои яйки
\begin{equation}
\frac{d}{dx}(2) = 0
\end{equation}
Вам пора задуматься об обучении на Физтехе
\begin{equation}
\frac{d}{dx}(1) = 0
\end{equation}
Блять завтра семестровая
\begin{equation}
\frac{d}{dx}(x) = 1
\end{equation}
Люблю кафедру общесоса
\begin{equation}
\frac{d}{dx}(\sin(x)) = \cos(x)
\end{equation}
Методом пристального взгляда
\begin{equation}
\frac{d}{dx}(1+\sin(x)) = \cos(x)
\end{equation}
Методом пристального взгляда
\begin{equation}
\frac{d}{dx}(\sqrt{1+\sin(x)}) = \frac{\cos(x)}{2 \cdot \sqrt{1+\sin(x)}}
\end{equation}
Методом пристального взгляда
\begin{equation}
\frac{d}{dx}(2 \cdot \sqrt{1+\sin(x)}) = \frac{\cos(x)}{2 \cdot \sqrt{1+\sin(x)}} \cdot 2
\end{equation}
Согласано предложению 1488 Знаменской Люмдмилы Николаевны
\begin{equation}
\frac{d}{dx}(\frac{\cos(x)}{2 \cdot \sqrt{1+\sin(x)}}) = P
\end{equation}
$$Где P = \frac{O}{{(2 \cdot \sqrt{1+\sin(x)})}^{2}}$$
$$Где O = -1 \cdot \sin(x) \cdot 2 \cdot \sqrt{1+\sin(x)}-\frac{\cos(x)}{2 \cdot \sqrt{1+\sin(x)}} \cdot 2 \cdot \cos(x)$$
Упростим

\begin{equation}
-1 \cdot 1 = -1
\end{equation}
Упростим

\begin{equation}
-1 = -1
\end{equation}
Каждый советский дошкольник знает
\begin{equation}
\frac{d}{dx}(\frac{1}{1+x} \cdot \frac{\cos(x)}{2 \cdot \sqrt{1+\sin(x)}}) = \frac{-1}{{(1+x)}^{2}} \cdot \frac{\cos(x)}{2 \cdot \sqrt{1+\sin(x)}}+P \cdot \frac{1}{1+x}
\end{equation}
$$Где P = \frac{O}{{(2 \cdot \sqrt{1+\sin(x)})}^{2}}$$
$$Где O = -1 \cdot \sin(x) \cdot 2 \cdot \sqrt{1+\sin(x)}-\frac{\cos(x)}{2 \cdot \sqrt{1+\sin(x)}} \cdot 2 \cdot \cos(x)$$
Методом пристального взгляда
\begin{equation}
\frac{d}{dx}(E) = Q
\end{equation}
$$Где E = D \cdot \ln(1+x)+\frac{1}{1+x} \cdot \frac{\cos(x)}{2 \cdot \sqrt{1+\sin(x)}}$$
$$Где D = \frac{C}{{(2 \cdot \sqrt{1+\sin(x)})}^{2}}$$
$$Где C = -1 \cdot \sin(x) \cdot 2 \cdot \sqrt{1+\sin(x)}-\frac{\cos(x)}{2 \cdot \sqrt{1+\sin(x)}} \cdot 2 \cdot \cos(x)$$
$$Где Q = N \cdot \ln(1+x)+\frac{1}{1+x} \cdot D+\frac{-1}{{(1+x)}^{2}} \cdot \frac{\cos(x)}{2 \cdot \sqrt{1+\sin(x)}}+P \cdot \frac{1}{1+x}$$
$$Где N = \frac{M}{{({(2 \cdot \sqrt{1+\sin(x)})}^{2})}^{2}}$$
$$Где M = I-L \cdot {(2 \cdot \sqrt{1+\sin(x)})}^{2}-2 \cdot 2 \cdot \sqrt{1+\sin(x)} \cdot C$$
$$Где I = \cos(x) \cdot -1 \cdot 2 \cdot \sqrt{1+\sin(x)}+\frac{\cos(x)}{2 \cdot \sqrt{1+\sin(x)}} \cdot 2 \cdot -1 \cdot \sin(x)$$
$$Где L = K \cdot 2 \cdot \cos(x)+-1 \cdot \sin(x) \cdot \frac{\cos(x)}{2 \cdot \sqrt{1+\sin(x)}} \cdot 2$$
$$Где K = \frac{J}{{(2 \cdot \sqrt{1+\sin(x)})}^{2}}$$
$$Где J = -1 \cdot \sin(x) \cdot 2 \cdot \sqrt{1+\sin(x)}-\frac{\cos(x)}{2 \cdot \sqrt{1+\sin(x)}} \cdot 2 \cdot \cos(x)$$
$$Где C = -1 \cdot \sin(x) \cdot 2 \cdot \sqrt{1+\sin(x)}-\frac{\cos(x)}{2 \cdot \sqrt{1+\sin(x)}} \cdot 2 \cdot \cos(x)$$
$$Где D = \frac{C}{{(2 \cdot \sqrt{1+\sin(x)})}^{2}}$$
$$Где C = -1 \cdot \sin(x) \cdot 2 \cdot \sqrt{1+\sin(x)}-\frac{\cos(x)}{2 \cdot \sqrt{1+\sin(x)}} \cdot 2 \cdot \cos(x)$$
$$Где P = \frac{O}{{(2 \cdot \sqrt{1+\sin(x)})}^{2}}$$
$$Где O = -1 \cdot \sin(x) \cdot 2 \cdot \sqrt{1+\sin(x)}-\frac{\cos(x)}{2 \cdot \sqrt{1+\sin(x)}} \cdot 2 \cdot \cos(x)$$
Очевидно, что
\begin{equation}
\frac{d}{dx}(B+E) = R
\end{equation}
$$Где B = \frac{-1}{{(1+x)}^{2}} \cdot \sqrt{1+\sin(x)}+\frac{\cos(x)}{2 \cdot \sqrt{1+\sin(x)}} \cdot \frac{1}{1+x}$$
$$Где E = D \cdot \ln(1+x)+\frac{1}{1+x} \cdot \frac{\cos(x)}{2 \cdot \sqrt{1+\sin(x)}}$$
$$Где D = \frac{C}{{(2 \cdot \sqrt{1+\sin(x)})}^{2}}$$
$$Где C = -1 \cdot \sin(x) \cdot 2 \cdot \sqrt{1+\sin(x)}-\frac{\cos(x)}{2 \cdot \sqrt{1+\sin(x)}} \cdot 2 \cdot \cos(x)$$
$$Где R = F+H \cdot \frac{1}{1+x}+\frac{-1}{{(1+x)}^{2}} \cdot \frac{\cos(x)}{2 \cdot \sqrt{1+\sin(x)}}+Q$$
$$Где F = \frac{-1 \cdot 2 \cdot 1+x \cdot -1}{{({(1+x)}^{2})}^{2}} \cdot \sqrt{1+\sin(x)}+\frac{\cos(x)}{2 \cdot \sqrt{1+\sin(x)}} \cdot \frac{-1}{{(1+x)}^{2}}$$
$$Где H = \frac{G}{{(2 \cdot \sqrt{1+\sin(x)})}^{2}}$$
$$Где G = -1 \cdot \sin(x) \cdot 2 \cdot \sqrt{1+\sin(x)}-\frac{\cos(x)}{2 \cdot \sqrt{1+\sin(x)}} \cdot 2 \cdot \cos(x)$$
$$Где Q = N \cdot \ln(1+x)+\frac{1}{1+x} \cdot D+\frac{-1}{{(1+x)}^{2}} \cdot \frac{\cos(x)}{2 \cdot \sqrt{1+\sin(x)}}+P \cdot \frac{1}{1+x}$$
$$Где N = \frac{M}{{({(2 \cdot \sqrt{1+\sin(x)})}^{2})}^{2}}$$
$$Где M = I-L \cdot {(2 \cdot \sqrt{1+\sin(x)})}^{2}-2 \cdot 2 \cdot \sqrt{1+\sin(x)} \cdot C$$
$$Где I = \cos(x) \cdot -1 \cdot 2 \cdot \sqrt{1+\sin(x)}+\frac{\cos(x)}{2 \cdot \sqrt{1+\sin(x)}} \cdot 2 \cdot -1 \cdot \sin(x)$$
$$Где L = K \cdot 2 \cdot \cos(x)+-1 \cdot \sin(x) \cdot \frac{\cos(x)}{2 \cdot \sqrt{1+\sin(x)}} \cdot 2$$
$$Где K = \frac{J}{{(2 \cdot \sqrt{1+\sin(x)})}^{2}}$$
$$Где J = -1 \cdot \sin(x) \cdot 2 \cdot \sqrt{1+\sin(x)}-\frac{\cos(x)}{2 \cdot \sqrt{1+\sin(x)}} \cdot 2 \cdot \cos(x)$$
$$Где C = -1 \cdot \sin(x) \cdot 2 \cdot \sqrt{1+\sin(x)}-\frac{\cos(x)}{2 \cdot \sqrt{1+\sin(x)}} \cdot 2 \cdot \cos(x)$$
$$Где D = \frac{C}{{(2 \cdot \sqrt{1+\sin(x)})}^{2}}$$
$$Где C = -1 \cdot \sin(x) \cdot 2 \cdot \sqrt{1+\sin(x)}-\frac{\cos(x)}{2 \cdot \sqrt{1+\sin(x)}} \cdot 2 \cdot \cos(x)$$
$$Где P = \frac{O}{{(2 \cdot \sqrt{1+\sin(x)})}^{2}}$$
$$Где O = -1 \cdot \sin(x) \cdot 2 \cdot \sqrt{1+\sin(x)}-\frac{\cos(x)}{2 \cdot \sqrt{1+\sin(x)}} \cdot 2 \cdot \cos(x)$$


Вот мы и посчитали производную. Кстати, уважаемая КВМ, пососите мои яйки.

\end{document}
